% options for packages loaded elsewhere
\PassOptionsToPackage{unicode=true,colorlinks=true,urlcolor=blue}{hyperref}
\PassOptionsToPackage{hyphens}{url}
\documentclass[a4paper,zihao=-4,notitlepage,twoside,openright]{ctexart}

\usepackage{ifxetex}
\ifxetex{}
\else
\errmessage{Must be built with XeLaTeX}
\fi

\input{../common/fonts.tex}
\input{../common/packages.tex}
\input{../common/setup.tex}

\title{COC模组设计之动机篇\\
{\footnotesize \url{http://wiki.cnmods.org/doku.php?id=user:\%E7\%AB\%A0\%E9\%B1\%BC:coc\%E6\%A8\%A1\%E7\%BB\%84\%E8\%AE\%BE\%E8\%AE\%A1\%E4\%B9\%8B\%E5\%8A\%A8\%E6\%9C\%BA&rev=1579669619}}}
\author{章鱼}
\date{2020/01/22}

\begin{document}
\maketitle

不出意外的话, COC模组设计应该是个系列文章,按照一定的顺序来谈一谈模组设计中需要注意的环节。

第一个也是最常出现问题的,那就是“动机”。

\section{何为动机}

所谓动机,就是调查员们因为何种原因而介入事件。动机有强有弱,而毫无动机是一个不利因素。我们首先来谈谈强弱动机的区别以及设置方法,至于特殊的无动机则在最后讲解。

\section{强动机与弱动机}

调查员们介入事件的原因就是动机,那么这个原因对调查员的强制力或者说是关联度,则是决定强弱动机的分野。举个简单的例子,如果调查员去某地旅游而遭遇了神秘事件,那么这就是个弱动机,他可以随时选择逃避危险,毕竟不是所有人都有一颗冒险的心。而同样的背景,调查员去旅游的途中感染上了某种不知名的病毒,而这种病毒正在小镇上肆虐,因此死去的人很多,这时候对调查员而言,他不解决这个病毒自己就有生命危险,这对他来说就是强动机。

同样的背景,我们还可以将剧情进一步复杂化,不牵涉病毒也没有传染病。我们的调查员在旅途中结识了一位新朋友,他们交谈甚欢。分道扬镳之后不久,一伙来历不明的人就开始追踪调查员,他们声称调查员身上有他们需要的一样东西,那个东西是那位新朋友交给调查员让他带走的(或者直接将调查员认为是新朋友的同伙)。那帮人绝对不会听调查员的解释,没有东西他们不会放过你,调查员不得不想办法摆脱或者解决这些人,他可能还要找到那位新朋友质问他到底怎么回事,而策划这一切的新朋友也许会在不远处窥探着,他也会伺机而动。

这个复杂的剧情可以被安排在旅途中的一座小镇,或者是某个偏僻的旅游目的地,让调查员无法从外界获得较多的援助。他们必须依靠自己或者当地少数的协助者,否则他们将被那伙人抓住,落入难以想象的困境。

让我们再做一次变化。同样的旅途,调查员想顺便拜访住在某个偏僻小镇的朋友(或者随便什么亲戚),然后当他抵达目的地会发现当地发生了一些事儿。

这里又有许多选择:

我们可以让小镇染上奇怪的病(这个关联类似第一个病毒的例子);也可以让那位朋友失踪并留下信息,他预料到了自己可能出事并将自己的财产包括这栋房子都赠予调查员(当然这并不是毫无代价的,或许房子里有问题,或许那些黑暗之物会窥伺这座房子等等),继承了这笔财产等于同时把麻烦也一并继承;又或者这位朋友热情的接待了调查员,但暗地里却计划着对他们不利,只为了完成他的邪恶阴谋;又或许小镇上开始流传一种奇怪的信仰,它正逐渐将镇民们非人化,这时候调查员发现自己无法离开小镇,他们不得不想办法应付这些问题。

你看,同样的一次旅途,我们可以为它添置一些信息或者限制,就会有一个颇具关联度的动机出现了,这种给调查员的行动增加变数的方式就是“加法”。 还有一种调整动机的方法叫做“减法”,与“加法”不同,它直接从调查员那里拿走什么,比如被窃的珍贵之物、失踪的亲人朋友、生命中曾犯下的错误、遭到诋毁的名誉等等。这种失去了什么必须将其追回的执着信念,也能成为调查员的强大动机。

在模组开始时,请尽量让调查员处于高关联度的情况下,这样他们才会有足够的动力去解决问题,而不是充当一个无关紧要的旁观者。

需要注意的是,由于一些特殊原因,目前最有强关联度的动机,依然是与调查员自身利益相关的东西,比如生命、健康、亲人、朋友、金钱、仇恨、报恩等等,其他的动机一般都视为弱动机。但在国外的模组中,这些也许会被视为强动机,比如荣誉、名誉、承诺、对弱者的怜悯和保护、好奇心、冒险精神等等。

这些强弱动机的分野不算特别明显,但广泛的说,国内的玩家一般很少会为了弱动机付出生命的代价。动机与面临的困难是成正比的,在较简单的事件中,完成难度不高没有生命危险,弱动机也不是不可以出现。但在复杂的事件里,有生命危险的同时,则强动机的驱动力更好,也更不容易让调查员跳车。

我们可以在诸多小说动画中看到,名侦探们都是遇到案子就扑上去的超级有表现欲和好奇心的家伙,除了这些“弱动机”之外,他们还有一个遇到困难也不能退缩的理由,“名侦探的信誉”,比如某个动不动就拿爷爷的名誉打赌的家伙。这一点即便是国内,也会有用武之地,某些职业特别在乎名誉,这关系着他们的职业生涯能否顺畅,所以他们对名誉的珍视超出其他人。

让我引用一下我之前写过的三种死亡的恐惧,分别是肉体死亡的恐惧、职业死亡的恐惧以及精神死亡的恐惧。这三种恐惧可以作为动机供选择,肉体的死亡无需多言;职业死亡即失去地位/失去职位/失去努力奋斗的一切,这是需要一些阅历方能理解的恐惧;最后一种是精神死亡,即肉体没有或很少受到伤害,但最重要的东西失去了,遭遇了无可挽回的变故,例如心爱的人、珍视的宝物、恪守的准则等等。

根据不同的玩家类型,灵活使用这些变数,让他们扮演的调查员陷入自己最不想看到的困境,逼迫他们去改变这一切。

这就是动机的作用。

\section{动机与跑团}

上一节说的是设计模组时如何安排动机,这一节要讲讲跑团时动机的作用。动机在故事开始时能给调查员一个介入事件的理由,让他们从某个起点开始,向预设的目的地前进。

如果没有动机,那么调查员将会陷入无目的的迷惑之中,这时守秘人(KP)只能把希望寄托在调查员本人的好奇心上了,等于故事已经脱离了守秘人的掌控,变成了守株待兔,这是比较糟糕的情况。更糟糕的可能是调查员一开始就偏离了主线,因为他们没有动机,不会有明确的目的和紧迫感,让一场好好的游戏滑向不可测的深渊。

动机除了能让调查员具有明确的目的和紧迫感之外,对于守秘人来说,动机也是能让守秘人掌控故事的法宝。任何事情都有一定的逻辑性,守秘人的故事也是如此,因此与故事配套的动机,能让调查员一开始就有针对性的目标,比如一起失窃案,他们可能会先去看看现场,查询一下失窃的物品情报,询问一下当事人等等,这些都是可预测的。调查员的行动可预测,这对于守秘人带团来说非常重要。只有可预测的行动,守秘人才能事先做好准备,不至于临时翻资料或者现编剧情,而这些突发情况很有可能导致不可预料的后果,比如现编的剧情与原本的剧情冲突,导致重大BUG,这是所有守秘人都不愿面对的情况。

守秘人的一项基本功是备团,所谓备团就是准备在带团过程中可能会用到的一切东西,包括熟悉剧本,推测玩家可能的行动,针对性的写出可能会用到的描述,根据动机推导出来的逻辑来安排故事线,将故事线中的节点厘清,准备好一些可能用到的支线剧情等等。但事实上,无论守秘人做了多么充分的准备,玩家总能给他无限的惊喜,导致有些东西用不上,而有些东西不够用。所以我才会非常强调动机在模组开头的重要性,这对于减少变数有着极大的帮助。

有了动机之后如何预测玩家行动,这方面有珠玉在前,我就不再赘述,对此有兴趣的读者可以看看舌尖上的狄拉克所写的科普文章《舌尖说:COC剧本如何预测玩家的行动》,这篇文章在魔都微博及公众号均有登载。

如果有守秘人(KP)拿到了一个没有动机或者动机不足的模组,那么作为守秘人你必须想办法让它成为强动机。就以最常见的无动机密室来说吧,通常的情况是调查员在睡梦中甚至正在平淡的生活中,突然被拉到密室里,然后就得面对接下来的种种怪事。我对此的评价是,如果不是密室,那么调查员多半会直接回家,谁有空陪你玩这种无聊游戏?

如何给这样的无动机密室增加强动机?复习一下第一节的内容,一个最简单的办法,生命倒计时,在《死亡铭刻》这个模组中,倒计时是通过腕表来提醒的,这就是一个成功的强动机。调查员时时刻刻面对着不断流逝的时间,这种压力会让他们充满战天斗地的勇气。

虽然生命倒计时是个不错的主意,但这毕竟是个补丁,调查员是被突然拉进密室的,他们本身对密室没有什么向往,甚至之前从未听说过。更加有建设性的方案是,在拉进密室之前稍微增加一点剧情,调查员因为某种几乎不可能实现的愿望(比如复活逝去的亲人、回到过去弥补错误、挣一大笔钱还债、找一个有钱又贤惠的老婆等等)而发现了进入密室的方法,为了实现愿望,他们不得不选择进入密室并努力完成里面的怪异谜题。这样的设计就让他们充满了主动性,而不是被动又茫然的在陌生的天花板下误打误撞。

\section{NPC的动机}

调查员的动机虽然常被忽视,但是还是有很多人知道这个东西的重要性,然而NPC的动机,则是个比调查员的动机更容易被忘却的存在。

这个题目的主要部分会放在后续的“人物篇”里详细讲解,因此这里就简单的谈一谈。

NPC在故事中的作用也是相当重要的。我们有一个误区,那就是喜欢将大多数的心思用在如何塑造怪物上,沉迷于营造大场面。但实际上,一个故事的核心始终是人,因为这个故事的读者是人,他们最关注的最有同理心的只会是人而不是别的东西。

一个成功的NPC会让玩家们印象深刻,而让NPC活起来的关键就是动机,这一点与调查员的动机并无两样。有了动机,NPC的行动也是可预测的,而不是作者笔下的提线木偶,它会有自己的灵魂有自己的习惯有自己的喜怒哀乐。

反派NPC有动机,同时与调查员的动机产生冲突,两者之间逐渐从了解到交锋,这段过程千变万化,可谓是模组中最重要的部分。因此一个合理的动机,对于NPC的塑造十分重要,这能让守秘人清楚的把握NPC的行动目标,该NPC会与调查员一样为了自己的动机而行动。

\section{动机的变化}

虽然初始动机通常能贯穿故事始终,但有时候也会有动机变化的情况产生。一般来说,只有当第一个动机被满足,第二个动机才会产生,但也有例外。比如在第一个动机尚未满足的时候,并列发生了一件事,使得调查员不得不着手应付,这就是第二动机。第二动机有可能与第一动机遥相呼应,解决了第二动机的事件对达成第一动机也有好处;也有可能互相矛盾,使得调查员必须做出选择,而无论他怎么选择都会带来遗憾。

一般情况下,我建议尽量不要让调查员拥有太多的动机,这会影响他的行动,让他变得不可捉摸不可预测。前面我们已经说过,调查员可预测的行动是顺利带团的基础之一。

\section{特殊的无动机}

前面我们说过,无动机对于模组来说是个不利因素,但有没有必须要开局无动机的模组呢?答案是有的。

一个简单的例子,比如失忆,调查员从故事开始就是失忆状态,他有可能记得或者不记得自己的名字,别的记忆也可有可无,因此他一开始便是迷茫的。 在这个特殊的开头中,我们必须尽快为调查员安排一个初始动机,让他能迅速进入我们预设的轨道,不至于产生太大的偏差。

安排初始动机的方式与故事有关,不同的故事有不同的方法,这方面有许多的电影电视剧可以供我们参考。比如说,一个在陌生的廉价小旅馆中醒来的调查员,他的身上也许会有什么笔记之类的东西提醒他第一步该做什么,这个笔记有可能是他失忆之前写下来的,也有可能是安排这一切的幕后黑手的误导。但不管怎么说,就好像在平静的水面丢下石子一样,只要有动作必然会留下痕迹,不管是什么引导着失忆的调查员,他最终都能发现些东西,这就是我们需要的。只要他动起来,我们就可以让他逐渐被卷入漩涡,直到最后真相大白或者一命归西。

\section{总结}

动机是引导调查员从故事起点出发的路标,它的作用非常重要,能为调查员指路,能为守秘人减轻工作。但与此同时,你并不能指望一个动机能让故事变得好看,能让游戏过程跌宕起伏,这与动机的强弱无关。

那它与什么有关?

这就是我们下一篇要讲的——悬念。

\end{document}
