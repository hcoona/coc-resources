% options for packages loaded elsewhere
\PassOptionsToPackage{unicode=true,colorlinks=true,urlcolor=blue}{hyperref}
\PassOptionsToPackage{hyphens}{url}
\documentclass[a4paper,zihao=-4,notitlepage,twoside,openright]{ctexart}

\usepackage{ifxetex}
\ifxetex{}
\else
\errmessage{Must be built with XeLaTeX}
\fi

\input{../common/fonts.tex}
\input{../common/packages.tex}
\input{../common/setup.tex}

\title{COC模组分析工具\\
{\footnotesize \url{http://wiki.cnmods.org/doku.php?id=user:\%E7\%AB\%A0\%E9\%B1\%BC:coc\%E6\%A8\%A1\%E7\%BB\%84\%E5\%88\%86\%E6\%9E\%90\%E5\%B7\%A5\%E5\%85\%B7&rev=1556163480}}}
\author{章鱼}
\date{2019/04/25}

\begin{document}
\maketitle

\section{前言}

创建点评俱乐部之初,我就粗略的弄了一个图,随便的列举了一下如何多角度对模组进行分析。但这个图中所列的内容过于简略,而且很容易跑偏,但当时我并没有把这个当做重心,于是一直就放在那里。

18年年末的时候,我困惑于如何管理团队,开始尝试自学管理方面的知识。到了11月中旬,我读到了一篇对李想的访谈,里面谈到了共识和系统管理及工具的话题,给我很大启发。

共识是我一开始就在点评俱乐部内部推行的东西,因为俱乐部是无偿义务劳动,那么大家能从这个活动当中得到什么就是我必须着重关注的。如果他觉得在这里得不到益处,自然就会离开,这也是我希望不断改进的地方。因此我将俱乐部的共识确定为“帮助他人,提升自我”,在分析及点评他人模组的过程中,一方面增宽见识面,一方面揣摩学习他人的长处,弥补自己的短处。从理想情况上来说,这个共识是可行的,但仍需实际运用来看效果。

但我忽视了一点,有了共识还不够,还需要一个系统工具,帮助所有人运用类似的方式对模组进行分析,并以此为基础进行讨论完善。同时,这个分析工具也能给所有人以启发和帮助,算是抛砖引玉吧。

\section{分析工具的定义}

分析工具自然是个工具,它的主要用途是帮助点评人以一定的角度来分析模组,它并不是万能的,最终成品的质量仍旧要看使用人的水平。分析工具能做的,就是让你不用花费额外的精力去归纳总结如何对模组进行拆分,只需要按照一定的方式去将模组拆分为更小的组件,对小的组件进行分析比对整个模组进行分析要简单许多。

下面就开始进入正题。

\section{六项分析法}

六项分析法是一整套对于模组剧本等带有文学性质的作品的分析方法,其中达到两项即可称为合格。一般应用最多的,也是作者们常常犯错的,就是第一项“动机”。我原本打算写个六部曲,第一篇“COC模组设计之动机篇”已经完稿,“悬念篇”仍在筹划。

六项分析法包括“动机”、“悬念”、“人物”、“气氛”、“主题”、“节奏”六个分析目标。下面就一一介绍,如何使用六项分析法来分析模组。

\subsection{动机}

动机可谓是COC模组的重灾区,有相当多的作者并不了解赋予动机的重要性。而且动机这个关键词,不光是针对调查员,同时也针对所有的NPC,每个出场人物都有自己的目的和手段,彼此汇聚才成为了一个鲜活的故事,这个部分也跟第三项“人物”有关,但我们还是主要说动机。

拿到一个模组,第一件事就是看动机,这应当是所有点评人及守秘人的共识。只有赋予了足够的动机,调查员才能全身心的投入这样一场也许事关生死的冒险,才能尽量减少各种跑团和带团中的问题。

模组阅读者必须首先问自己一个问题,这个模组里调查员如何介入故事剧情?他们是因为什么原因介入事件?这个原因合理吗?有什么阻碍他们深入剧情的东西吗?搞清楚了这些问题,阅读者自然就会知道,这个故事能不能在开头就平稳的进行下去,而不是开局就翻车。

有些模组在初始时不能赋予动机,但通常也会在开局不久后便补上这一课。以我自己的模组《自走的人偶》为例,因为是失忆,所以调查员醒来时不知道该做什么,只知道自己在陌生的房间里。因此我必须尽快赋予动机,而我也尽量这么做了,但显然任何情节都是有被忽略的可能性,因此一旦有调查员忽略了赋予动机的情节,他很可能就会落到破局失败的境地。这也是我尝试了无动机密室型模组之后,再次确认的该类型模组无法避免的弊病。

还有些模组会在中途切换动机,也就是一环套一环,这样的模组与之前的模组并没有什么区别,在切换动机的同时,同样可以按照刚才的方式进行分析。切换的动机合理吗?调查员有足够的动机去继续进行调查吗?有什么阻碍?这样的分析总是能有效果的。

除了初始单动机之外,还有一些较为复杂的模组选择了多重动机。多重动机与单动机的分析区别在于,阅读者要抓住多重动机中最为重要的那个,然后依然使用单动机的分析方法进行分析。比如某个模组的调查员有三个动机分别是“复仇”、“调查秘密”、“取得宝藏”,那么一般来说“复仇”应该是主要动机,我们就暂且只抓主要动机,分析完了主要动机再来分析次要动机,把这些动机分开来看,化整为零,这样处理会简单一些。 除了调查员的动机之外,NPC的动机也很重要,特别是剧情中的主要人物,他们的主要目标是什么?为什么这个目标对他这么重要?为了实现这个目标,他愿意做出哪些牺牲?哪些东西会让他放弃目标?

有一套很有用的自我分析方法,也可以用在NPC上,这个方法有三问,分别是“我想要什么?”、“我要放弃什么?”、“我要承担哪些责任?”。前两条自问是关于订立心理警戒线,最后一条是有关实现方法。做好了这些分析,一旦遇到损失真实发生,也就不会惊慌失措。

对于我们的NPC来说,比方一个大反派,他的目的是什么?他能从中得到什么益处?会因此损失什么?他愿意付出什么代价?他如何去实现这个目的?这套实现方法符合他的设定吗?这些问题的最终答案都会指向我们所想知道的东西。

\subsection{悬念}

文似看山不喜平,这也可以作为对模组的要求。在确认了动机没有问题之后,阅读者的第二个重点便应当是“悬念”。动机提供了调查员介入事件的理由和执着度,但故事本身的吸引力则要依靠悬念来营造。

模组阅读者应当问自己:“这个故事有悬念吗?足够吸引人吗?有让人意外的转折吗?”并在接下来的阅读分析中逐个解答这些问题,那么你对这个模组就会有更深的了解。

通常来说,故事的开头会有一个小的悬念引入事件,比如调查员们受雇去调查一桩疑似婚外情的案件(天堂之水),那么这个婚外情事件是真是假便是一个小悬念,在挖掘事件真相的过程中便可再次引入其他的悬念。那么现在你应该明白悬念是什么了,悬念就是一种预兆或者一种未知的可能性,这给了玩家们遐想空间,让他们对接下来的事情抱有期待,而不是按部就班的完成确定的工作。

最重要的就是期待感,玩家们会对这个事情感兴趣,那么悬念就是成功的,否则就等于失败。阅读模组时,判断悬念是否成功需要一定的经验,但阅读者完全可以单纯以自己的角度来感觉这个悬念对你是否有吸引力,通常来说这样的判断并不会太离谱,因为大多数人类的感知是相似的。

从悬念的角度来分析模组,可以按照悬念出现的顺序进行分析,也可以从重要程度进行拆分,如果你有更棒的主意,还可以对现有的悬念提出质疑。保持悬念最重要的一件事,就是让玩家感到意外,而不是“我就知道是这样”。在分析模组的时候,能否保持让人意外的悬念,也是重要的一部分。

\subsection{人物}

人物的塑造是小说的重中之重,模组也属于文学作品,当然也应该向这个方向努力。一个鲜活的人物便足以带动一整个模组的气氛与主题,这一点可以参考47的《少女风情》,当然他的模组里其他NPC角色也有闪光之处,但女主角的光彩是别人遮掩不住的。

我们分析模组的人物,应该首先自问几个问题:“这个模组里有让你印象深刻的人物吗?” 、“这个(些)人物给你最深印象的情节有哪些?” 、“作者使用了什么手法来突出表现这个(些)人物?”、“这个手法效果如何?你有什么感想或者建议吗?”、“在实际带团运用中如何来表现这个(些)人物的特质从而让玩家能体会到这些特质?”

或者还有更多的类似问题,关于人物这方面,可写的内容非常之多,但分析点评需要抓住重点,因此如果你打算以人物入手进行分析,那么你就要抓住故事里的灵魂人物,并以他为中心来进行分析,包括正面分析和侧面烘托。这里的侧面烘托则是以NPC的角度进行的,这等于是还原并梳理人物关系。这里我要推荐一下,猪头教授的NPC人物看法表格,将NPC填入表格的X轴和Y轴,并在交汇处写上彼此的印象,这是很直观的一种表达方式。

我们还是以47的《少女风情》为例,他的女主角百合子,沉浸在幻想之中无法自拔,当她与蝶结合得到力量之后,连调查员也会被玩弄于幻觉之中,甚至到了最后已经没人能分清什么才是真什么才是假。

百合子这个人物充分的利用了意外和悬念来颠覆人物形象,从开始时的贤淑女子,到中途产生怀疑,最后真假难辨。大部分情节设置,都是围绕着人物形象来进行的,就是这样才塑造出了这个让人印象深刻的女主角形象。47在塑造人物时很少使用旁白式的描写,绝不盖棺定论,而是以各种NPC的主观陈述来“片面”的描述女主角,这样的方式自然而然的产生了混淆视线的效果。这与二十四节气设定集中的“盲人摸象”原则无独有偶,都是利用主观的方式进行描绘,从而达成一种“雾里看花”的效果。同时,这种描述方式,又能引起读者的共鸣,人类就是这样的生物,自己看到便信以为真,但实际上也许没人能看清那到底是什么。这一点便拔高到了“主题”的部分,也就是所谓的“立意”,也可以理解为“作者到底想告诉你什么”。

当然,这一点也许就像是语文老师布置的“主题思想”,作者本来没有这样的想法,硬要总结出一个想法。不过这没有关系,我们能从作品得到的,就是我们自己的,与作者已经没有关系了不是么。

\subsection{氛围}

感受模组的氛围并不需要什么高深的学问,这种对于气氛的感知能力存在于每个人的身体里。我们都有这样的经历,只要进屋甚至不需要多长时间,就能感受到屋内的气氛如何,是兴奋激荡还是开心快乐或者暗流涌动。那一瞬间的难明感觉,是无法用语言表述的。

但是分析氛围就是另一回事了,它仍然需要进行逻辑思维和自问设答。对于COC模组来说,最合适的当然是未知的恐惧,惊悚恐怖这些氛围,但模组中如何表现这些东西,是点评人需要解剖出来的。

首先,问自己“模组中氛围表现明显吗?我是否感受到了某种氛围?”,然后是“此种氛围的表现形式是什么?效果好吗?可否改进?”,最后是“我自己能从中得到什么灵感或者帮助?”

对于氛围的分析,我们还可以进行技术细节上的列举和细分。比如,一个未知的恐惧是如何呈现的,它的要点是什么,最重要的点和最不重要的点有什么本质上的区别?

也可以从另一个方面去分析,比如你是如何感知到氛围存在的,氛围是如何对你起作用,你又如何将这种感知具现化到文字上。

举个例子,魔都模组《正西方》,它所营造的恐惧氛围,就在于西方未知的存在,所有去过那里的人都变了,这种改变是可见的,但那个存在到底是什么,没有人知道。它的表现方式是,能明确观察到改变,但导致改变的本源是无法观察的,于是这种改变越恐怖,其背后的本源就更加恐怖。以可见推出不可见,人类自己的想象足以营造出超越极限的东西。它营造氛围的本质就是,不直接观察和定义恐怖本源,而是着力于表现本源映射出的表象,通过表象的异常来衬托本源的恐怖。简单的说,也可以归类为对比的手法。

这样的分析从手法上进行,并最终总结出要点和概括,能够为其他人带来灵感或借鉴。

\subsection{主题}

主题即身为作者你想展现给读者的是什么?比如原著爱手艺大师表达的主题就是“无知是福”、“人类不应该探求禁忌的知识”之类的思想。这类作品中,探索和发掘禁忌黑暗之物的人通常不会有好下场,这就是剧情为主题而服务。

可以说,作者的主题思想就是大脑,而剧情就是身体,只能大脑指挥身体,而不能让身体指挥大脑。有一些作者的作品里没有主题思想,这也是客观存在的,他/她只是写一个故事,并没有往深处想表达什么,也许他/她的故事里有什么意义,但并不明显,也不重要。

主题并不是不可或缺的,这一点与人体并不一样,人体没有大脑会死,模组没有主题也能玩。我们分析主题的时候,当然不能无中生有,还是要分析主题的产生和它的实现手法,即故事如何围绕主题进行阐述和表现。

对于作者的建议方面,可以谈谈主题的凝练和升华。对于没有主题的模组,也可以谈谈如何在现有剧情上附加主题。

有些主题是依靠人物形象和遭遇表现的,比如《少女风情》,又比如《土豆村》和《追书人》。这些主题的分析难免会与人物分析产生关系,可以把着重点放在人物遭遇上,从遭遇和反应来分析人物性格及表达的主题。

\subsection{节奏}

关于节奏,我自己也在摸索阶段,说不出什么道理,只有一些个人体会。所谓节奏,就是“起承转合”的度,这个度需要经验和一些天赋去把握。什么时候该平淡,什么时候该紧张,什么时候可以渲染,什么时候应该搞一次小摩擦,这都是与心理学有一些关联的东西。

一个好的节奏,能让玩家始终把关注度保持在一个水平之上,比如开局的渐入佳境,很快迎来第一个紧张的点;前半局顺利的调查突然遇到挫折,绝处逢生;后半局克服困难,迎难而上,最终探明真相解决事件。

这样看起来,似乎与悬念有许多共同之处,但悬念与节奏还是有着细微的不同的。悬念更倾向于片段的疑问与答案,而节奏则是大局观上的抑扬顿挫。

分析模组的节奏,首先要对模组进行肢解,将其拆分为平和高两个部分,平部是代表平缓的剧情,如调查、取证、查阅、赶路等等;而高部则是较为激动和紧张的剧情,如交涉、战斗、解谜、斗智等等。

简单的说,调查员的个人行为与外界关联较少时,一般属于平部;与外界交互的越深入,就越倾向高部。高平结合,安排得当,就是一个良好的节奏。

具体的分析,这里就无法给出答案了,需要点评人进行独立的思考与定论。每个人的看法都有不同,同一个事物得出不同的结论也属正常。

\end{document}
