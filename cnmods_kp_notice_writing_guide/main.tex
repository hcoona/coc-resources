% options for packages loaded elsewhere
\PassOptionsToPackage{unicode=true,colorlinks=true,urlcolor=blue}{hyperref}
\PassOptionsToPackage{hyphens}{url}
\documentclass[a4paper,zihao=-4,notitlepage,twoside,openright]{ctexart}

\usepackage{ifxetex}
\ifxetex{}
\else
\errmessage{Must be built with XeLaTeX}
\fi

\usepackage{amssymb,amsmath}
\usepackage{fontspec}
\usepackage{fourier}
\setmonofont{iosevka-type-slab-regular}[
  Path=../common/iosevka-type-slab/,
  Extension=.ttf,
  BoldFont=iosevka-type-slab-bold,
  ItalicFont=iosevka-type-slab-italic,
  BoldItalicFont=iosevka-type-slab-bolditalic,
  Scale=MatchLowercase,
]

% Math
\usepackage[binary-units]{siunitx}

\usepackage{caption}
\usepackage{authblk}
\usepackage{enumitem}
\usepackage{footnote}

% Table
\usepackage{tabu}
\usepackage{longtable}
\usepackage{booktabs}
\usepackage{multirow}

% Verbatim & Source code
\usepackage{fancyvrb}

% Beauty
\usepackage[protrusion]{microtype}
\usepackage[all]{nowidow}
\usepackage{upquote}
\usepackage{parskip}
\usepackage[strict]{changepage}

\usepackage{hyperref}

% Graph
\usepackage{graphicx}
\usepackage{grffile}
\usepackage{tikz}

\hypersetup{
  bookmarksnumbered,
  pdfborder={0 0 0},
  pdfpagemode=UseNone,
  pdfstartview=FitH,
  breaklinks=true}
\urlstyle{same}  % don't use monospace font for urls

\usetikzlibrary{arrows.meta,calc,shapes.geometric,shapes.misc}

\setminted{
  autogobble,
  breakbytokenanywhere,
  breaklines,
  fontsize=\footnotesize,
}
\setmintedinline{
  autogobble,
  breakbytokenanywhere,
  breaklines,
  fontsize=\footnotesize,
}

\makeatletter
\def\maxwidth{\ifdim\Gin@nat@width>\linewidth\linewidth\else\Gin@nat@width\fi}
\def\maxheight{\ifdim\Gin@nat@height>\textheight\textheight\else\Gin@nat@height\fi}
\makeatother

% Scale images if necessary, so that they will not overflow the page
% margins by default, and it is still possible to overwrite the defaults
% using explicit options in \includegraphics[width, height, ...]{}
\setkeys{Gin}{width=\maxwidth,height=\maxheight,keepaspectratio}
\setlength{\emergencystretch}{3em}  % prevent overfull lines
\setcounter{secnumdepth}{3}

% Redefines (sub)paragraphs to behave more like sections
\ifx\paragraph\undefined\else
\let\oldparagraph\paragraph{}
\renewcommand{\paragraph}[1]{\oldparagraph{#1}\mbox{}}
\fi
\ifx\subparagraph\undefined\else
\let\oldsubparagraph\subparagraph{}
\renewcommand{\subparagraph}[1]{\oldsubparagraph{#1}\mbox{}}
\fi

% set default figure placement to htbp
\makeatletter
\def\fps@figure{htbp}
\makeatother


\title{守秘人须知写作指南\\
{\footnotesize \url{http://wiki.cnmods.org/doku.php?id=user:\%E7\%AB\%A0\%E9\%B1\%BC:kp\%E9\%A1\%BB\%E7\%9F\%A5}}}
\author{章鱼}
\date{2019/04/27}

\begin{document}
\maketitle

\section{什么是守秘人须知}

守秘人须知(KP须知)是包含在模组中的一个部分,这个部分一般位于模组头部,与版权信息通常是邻居。以我个人看过的千把个模组来说,守秘人须知写的好非常少,大部分的守秘人须知并不完整,通常只把这个模组是个什么故事说清楚了,有的甚至连故事都没提,只是讲了讲前情提要,具体的内容一概没有。

\section{守秘人须知的意义和作用}

那么守秘人须知到底是干什么用的呢?有人可能会回答就是把模组里发生的故事前因后果说清楚;还有人可能会说要把幕后隐藏的信息交代一下,对读模组的KP带团有所帮助。这些都说的对,但都不完整。

守秘人须知,顾名思义就是给守秘人看的,而且是这个模组里对守秘人最重要的部分。一个好的守秘人须知,让读者(也就是KP)看完之后,立刻就能了解这个模组讲了一个什么故事,反派是如何计划及行动的,有哪些独特的设定和房规,具体的游戏玩法是什么。在清楚明了这些信息之后,守秘人可以根据自身条件对该模组进行取舍,能不能带?怎么带?怎么进行修改以适合自己的长处?怎么修改以配合玩家的风格?以及需要做哪些准备等等。

简单的一句话总结,看完守秘人须知之后,读者(KP)就知道该如何备团。这是最基础的要求,可以节约大量的时间和精力,读者则不需要看完整个模组就能知道他面对的是什么。从开始就能以上帝视角明晰所有的情况和细节,这时候再去读模组的正文,必然是事半功倍。

\section{守秘人须知的组成部分}

那么守秘人须知到底该怎么写?包含有哪些成分呢?下面我就逐一说明。

\subsection{模组概述}

守秘人须知的第一部分就是模组概述,这个部分就像是内容简介又像是论文的概述,要准确而简略的将模组进行介绍。这个部分也包括有两个内容,一个是模组属性,一个是内容简述。

\subsection{模组属性}

模组属性是指与模组有关的信息,比如模组的背景年代、使用的何版规则及扩展、作者认为适合模组的游戏人数等。在这里我必须强调的是,车卡建议和技能推荐之类的活是KP的,除非是有特殊需要,不然作者不应该写这类的内容。

\subsection{内容简述}

内容简述是指作者就模组的大体内容进行简单的描述,给读者一个初步的印象。比如我现在要给我自己的模组《血脉的呼唤》写一个简述,那我会写:“这是二十四节气系列模组之一,讲述了由江城大学外籍教授狄拉克带队进行的一次神农架野外考察所引出的一系列事件。该故事主线比较清晰,有几个清晰的节点,便于KP调整,部分细节可能需要填充,对于调查员的引导和重要线索的给出是重点。”

\subsection{故事梗概}

故事梗概与内容简述十分相似,但要详细的多。作者在这一段要把故事的起因、经过和结果都简练写出来,让读者(KP)能迅速了解这是个什么故事,有哪些重要的转折,结局大概是什么样的。

\subsection{出场人物}

出场人物介绍,这个部分也可以放在附录,这是唯一一个不是必须在守秘人须知出现的内容。但有出场人物介绍能更利于读者(KP)扮演,从而给玩家更好的体验。一般来说,作者会更了解自己笔下的人物细节,把这些细节写出来,对带团肯定是有帮助的。另外还有一点,有些作者经常忘记给出场人物拟定数值,这会给读者(KP)造成很多不必要的负担。出场人物(包括怪物)通常应该给出详细的数据和文字说明,或者进一步给出扮演建议或者行动方针。在这里我要推荐一下胖绅士的NPC态度表格法(范例见附录),把NPC对彼此的态度用表格的形式表现出来。

\subsection{幕后阴谋}

这一部分是玩家扮演的调查员有可能接触不到的,但读者(KP)必须了解,所以作者要在守秘人须知这个环节里详细的交代清楚。这个部分包括反派的动机和目标、反派的行动方针、行动计划以及在模组的故事进行中可能出现的组织或NPC的行动等等。最好能把模组中所有出现的NPC(包括组织和怪物)的动机都列出来,这样对于KP来说一目了然,带团自然更方便。有些故事里可能没有反派,那么作者在下一个环节,也就是主要玩法里就要多说几句了。

\subsection{主要玩法}

这个环节主要是介绍模组的运行规律,还是以《血脉的呼唤》为例,我会这样介绍:“这是一个场景推进式的模组,每个场景都列出了可能发现的线索和必须发现的线索,没有取得必须的线索会导致调查员无法推进剧情,因此KP务必确保这些线索被合适的发放到调查员手中。游戏中的变数来自于调查员们的探索行动是否被发现,如果被发现,他们的目标可能会警觉起来并提早行动,对于这一点KP要做好充足的准备。测试团的情况表明,调查员们可能会对大山中的镇子感兴趣,如果你希望他们留在旅馆,要提前想个好办法。”

如果故事里没有反派,那么调查员们可能要面对的是紧迫的时间或者自然的威胁,这个故事也许需要他们同心协力共渡难关,也可能是勾心斗角明枪暗箭。但不管怎样,作者都应该尽力描述自己预想中的游戏方式,提醒作者注意那些容易被忽略的地方。

就像我刚才所说,在这个环节里,作者要尽可能详细的给读者(KP)解释模组中的难点和要点,尽量让作者的预想能较好的实现,这可能不是一两句话能解决的事,请尽力而为吧。

\section*{附录}

图片由胖绅士提供:

\includegraphics[width=\columnwidth]{NPC.jpg}

\end{document}
