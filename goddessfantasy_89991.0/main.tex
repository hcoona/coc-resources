% options for packages loaded elsewhere
\PassOptionsToPackage{unicode=true,colorlinks=true,urlcolor=blue}{hyperref}
\PassOptionsToPackage{hyphens}{url}
\documentclass[a4paper,zihao=-4,notitlepage,twoside,openright]{ctexart}

\usepackage{ifxetex}
\ifxetex{}
\else
\errmessage{Must be built with XeLaTeX}
\fi

\input{../common/fonts.tex}
\input{../common/packages.tex}
\input{../common/setup.tex}

\title{[杂谈]如何从无到有建构一个探索调查型剧本?\\
{\footnotesize \url{http://www.goddessfantasy.net/bbs/index.php?topic=89991.0}}}
\author{Shinohara}
\date{2017-03-03}

\begin{document}
\maketitle

不限于特定系统,要从无到有建构一个探索调查型剧本,原则上有两个方向。
“什么发生了?”跟“发生了什么?”以及“有什么发生?”
由内而外或是由外而内,由根源到细节,或是从细节还原到根源。

假设基本的核心是“封印破了而发生了不好的事”,来一个个补完细节。封印破了,那原本封印的是什么?谁封印的?被封印的东西是从何而来的?什么时候封印的?什么时候封印破了,又是谁为了什么而打破的。

这是从原始的核心开始发想。

而从细节还原到核心,可能是一个片段或画面,也许在写剧本的最初就有一个想像中的场景,譬如说在废弃的油轮中拿着手电筒探险,或是在迷雾环绕的小村中遇见了古代的村人,有了概念印象后,再替这些印象找理由,一个个把细节与理由补充完整。

当对事件的核心来龙去脉有了想法后,要决定展露比例,有多少现象与细节是外显可被玩家直接知道的,有多少现象与细节是玩家必须达成某些事才能知晓或碰见的。

从整个事件中找出与玩家人物匹配的冒险引鈎,在必要时甚至是更动事件的内部细节,以让整个事件更有一体性而不突兀。在事件整个设计完,决定好玩家角色的切入点后,可以添加一些旁枝末节的调味细节,以及用来补破网或是添加趣味的要素。不需要在一开始写剧本时就老老实实的白纸黑字规划人物设定、重要场景、故事设定、导入和结局等等,保持一些弹性。

在觉得太空洞的地方填入东西,在觉得会不知所措的地方放进线索,不需要死硬抓着说某某线索一定是在某个地方,而是弹性地在需要时合理地出现在玩家所找的地方,这样会让焦头烂额的玩家得到成就感,而更积极去进行下一个动作。

不需要死抓着预设的规划,如果玩家的推论比你所拟定的故事还要合理精彩,不要害怕去更动设定,但要切记,鲁莽的更动有可能会造成故事前后不一牛头不对马嘴,因此要谨慎小心的处理,不要造成剧本上的矛盾。

将想要的场景与要素一股脑的全放进剧本中是很好,但也要考虑到墙上的枪理论,如果这关系不大,那就不要放进去误导玩家,又或者说,放进去误导可以,但总是要有所影响,不要追到最后是一无所获的死路,这会让人很沮丧,即使最后打开保险箱放的并不是关键线索,至少也放一把左轮让玩家不会什么都没有。

因此在构成剧本时,要考虑每个细节的必要性。当然,也许玩家会感觉放进来的东西一定是有用的,因此看到每个东西就紧抓着不放,但这总比消极不做事好。

关键的NPC必须要积极,或者说,让积极的NPC当作关键,否则就是在剧本设计上有很明确的指引玩家去找他麻烦,如果一没暗示,二这NPC只是低调地坐在酒馆阴暗处,那么冲过头的玩家很容易会忽略而错失线索。

当然这并不是说一定要给玩家很明显的指引,只不过如果把剧本中的关键线索藏的太深,就要有玩家遍寻不着的心理准备,这不是玩家太笨,而是单纯的资讯与观点落差,在全知观点的主持人眼中,线索永远都是看起来很清晰直接,但对玩家可不然。

在剧本的设计上,最好设计有时间性的事件发动,不见得是毁天灭地的大灾难,但是到某个时间点会发生某事,可以让玩家有风雨欲来的时间流动感,也有威胁步步进逼的感觉,而不是好像瞎转什么事都没发生。而这种时间性的事件也能配合玩家的调查进度来释出线索,当玩家陷入死局时,某些破冰性的事件可以让气氛改变,而重启调查或是指出新的方向。

这也能当作是剧本的时程表,当完成了某些事或是到了某个时间,整个剧本的环境就会进入下一个阶段。

剧本可以处于开放区域,但有几个要点,一个是你必须保证你的玩家们不会蓄意唱反调,不管手上拿到的线索跟这个区域有千丝万缕的关系,就是要跑去其他地方,另外一个就是设计上要环环相扣,让玩家们舍不得离开这个地方。当玩家角色在某个区域瞎转都没半点进展时,会想悍然舍弃跑去其他地方挖东西是很正常的,不是抛点饵料下去,就是安排指示往某处去。

当然这也要与玩家们有默契,如果玩家角色是抱着消极被动的保命心态,那毫无所获是很正常的。

一个思考误区是,有的剧本设计者会觉得这是一个你与玩家的智力挑战,看玩家能不能破解你的谜题,这是错误的想法。主持人与玩家并不是对抗性的,当你有这个心思,那玩家一心就要拆你的台也不是怪事了;准确来说,剧本是一个阶段展示故事的方式,在与玩家的默契下一起发现你规画中或是不在你规画中的戏剧性,也就是一个互相激发创意的共演。而对抗性往往只会有不快的结果。

为何不能以对抗性的心态来设计剧本,因为在游戏的过程中,会产生资讯落差与认知落差,这点并不是谁的错,而是每个不同经历的人,会以不同的角度与知识观点去理解同一件事,有些你看来根本就是毫无掩饰理所当然的东西,对于玩家来说是毫无头绪,而即使是简单的事物,也会因为认知落差的关系而产生误解,进而发生阻碍。

当主持人抱持着想当然尔的全知角度,就会觉得玩家为何又傻又笨不会发现疑点,但很可能这疑点在其他人的理解中根本不存在。

会想要建构一个新剧本,通常是因为得到零碎的灵感,而想将其完成为一个剧本,这个灵感可能是一个核心来源,例如说某个旧神的影响,或是一个概念场景,来源可能是昨天看到的一条有趣的新闻或是一幕戏剧性的场景,又或是某个在其他团中未能完成的怨念,想要发挥出来。

将这个灵感发展为完整全面性的故事,就成为剧本。不过要注意的是,只是为了自己某个恶趣味而写出剧本,其他玩家可能会不买帐,戒之。

接着用随机安价的例子来示范大概的做法。

核心来源:首先要决定事件核心。
这种探索解谜型剧本的建构方式不限于特定系统,不过既然原来的问题问的是克苏鲁的呼唤,那就以此来当作例子。
打开Google输入“克苏鲁神话”搜寻,点入wiki页面,从外神中随便选一个,为了减少再搜寻资料的麻烦,选择有直连页面的“伊德海拉”。

\url{https://zh.wikipedia.org/wiki/%E4%BC%8A%E5%BE%B7%E6%B5%B7%E6%8B%89}

伊德海拉(Yidhra)被称为梦之女巫(The Dream Witch),通常以一个年轻迷人的人类女性的形象出现,但她的外表可以变化。
伊德海拉自第一种微生物出现后便在地球上长存了(据传是乌波•萨斯拉通过裂殖产生的第一个个体)。

由于个人不喜欢拿邪神本体到处跑,而且有太强大的幕后黑手对剧本不是好事,太过强大到撂不倒动不了随便就被压制,对玩家来说很有挫折感,丧失了挑战性,而放水让玩家压制了邪神本体,又更荒诞不合理。

因此就把幕后黑手定为伊德海拉的其中一个微弱分裂体,并且有一群信奉她的狂信者。这就是克苏鲁神话故事中,标准的“小村庄与黑手及隐藏的邪教信徒们”故事的核心。

决定核心的来源/动机/目标/外显/影响。
来源是指这核心从何而来,是古老之前就在此处,还是曾经被封印然后放出来,或是旅行到此定居,因为某种目的而被制造等等。在此,因为伊德海拉有微生物、分裂等等的特征,暂定为是某个海边村庄的女子饮用了恰好有伊德海拉碎块的水,而被侵蚀附身。至于说伊德海拉碎块是从何而来,是在哪被击败而漂流至此,那就是另外一个故事了。

动机,其实常常搞不清楚克苏鲁神话中各种古神外神的动机,超越凡智的存在不是我们可以揣测的,总之大概设定一个简单的动机,就是存在的最大核心本能─继续存在下去;为了让剧本更单纯,将这边的伊德海拉分裂体设定为“被破坏而残存的碎片,因为神秘性的破坏之故,只剩下本能与些许记忆,无法与其他分裂的意识相连接”,简单的说就是之前的本体被魔法打到记忆丧失,然后现在只剩断片就是。

目标,与动机相关,为了生存以及受本性影响,核心将此村庄视为自己的领域不容侵犯,利用了其梦境的能力盘据此地,而村民们在长久数代的洗脑下,造出了一个新宗教,只要向其祈愿就能获得安眠与美梦,因此村民们都过得很幸福,即使是生活再困苦再严苛,都能在梦境中获得补偿,甚至认为白日肉体的重劳动是一种祝福,能让自己在夜晚更快更顺利的进入梦境。
在这种情况下,村民排斥外来者进入分与“神恩”,而这也是伊德海拉所授意,避免被干扰。

该存在的目标是,盘据此地经营,并且在蓄积足够力量后扩大影响,以便能探查出自己的来源,或是找出是什么摧毁了之前的自己,以及试图与其他分裂接上线。
确立了动机目标后,就能以此来设计该有的应对与态度方针。

外显是表示在这些影响之下,玩家角色们到达此地时会看到的现象。村民们笑容可掬亲切对人,但又有种疏离感,对生活缺乏梦想与积极动力(因为梦想都会在梦中完成),甚至会展现出类似苦行的状况。

村庄内的教堂信奉的是奇怪的神祉,但是细究其教义似乎也不是什么很扭曲奇怪的邪教,为了放入疑点,可以设计该教派有一个内部组织,只有少数的高阶信奉者才能接触,这就是真正的邪教组织,而普通村民所知道的只是外壳的掩护教义。

影响则是分为范围影响与个体影响,也是表现在外的状态,以让玩家抽丝剥茧探查的疑点,譬如说可能某个电视节目来报导说“最快乐的渔村”,或是从邻近的村庄因为要拓点的关系,在梦中开始受到感召。

而最重要的是,要依据这些现象来替每个玩家角色来设计冒险引勾。

\begin{description}
\item[玩家A] 某个研究宗教的朋友发现这个村庄有独特的宗教,来此研究后一去不返,后来到这才发现他已经定居在此,向玩家述说这个村庄的美好,但同时吃着难以下咽的粗糙食物。
\item[玩家B] 另一个邪教的信徒,认为此地扩大的信仰侵犯到自己的宗教,特地前来探查。
\item[玩家C] 调查员学徒,从前辈的纪载中知道数十年前曾经驱除掉这个外神,但是怀疑此村庄受到残留的影响。
\end{description}

就如同上面所说,从整个事件中找出与玩家人物匹配的冒险引鈎,在必要时甚至是更动事件的内部细节,以让整个事件更有一体性而不突兀。
设计得好的冒险引勾,能够让玩家角色更有积极性,也能更投入故事中,这不再是“别人的事情”,而是跟角色有切身关系。玩家也比较不会玩一玩觉得没头绪,就喊着要离开这边去其他地方乱挖。

在撰写剧本时先预设好几个动机与接入口,这样在到时候游戏时玩家才会更积极的投入故事。无论是物品或是人物都好,要让玩家感觉到这与自己的角色切身相关,而又要同时查觉到一些可以开始动手调查的疑点。

为了游戏的乐趣,可以设计不同的挑战,或是以不同方式解决会有不同的结果,例如说让每个角色都有发挥的场面,可以安排社交的、智力的,或是体能上的挑战。而同样的非玩家角色,用劝说的或是威胁的,会有两种不同的说词。

如果想要设计出一个“拟似开放性”的剧本,那么线索接入口就要设计许多条线,就像是拼图一样,从每个接入口开始探索而得到的“真相”,会变成一片片的拼图,逐渐拼凑出事件的全貌,而被忽略或是探所失败的线索也无妨,可以从其他的拼图找出那块缺失的可能性,又或者就这样留下迷团也很有意思。

设计上可以安排每条线索都能获得一些助益,而在结局能派得上用场,例如说某个回避探查的护符,一把轰破阻碍的散弹枪,或是取得某个人物的信任能通过密门等。

等到这些基础大纲都设定好,就可以开始把一些细节填入,虽然说设定成对某个特定的非玩家人物说出特定的关键字才能获得资讯看似严谨仔细,但是在游戏中并不会如此顺遂,因此还是留下些许弹性才好,另外为了避免玩家过于被动或是摸不着头绪,也能放入协助者,例如说天生对梦境有抵抗力的小男孩村民,而其他村民只觉得他是个叛逆且爱说谎的小孩。

当大致上完成剧本后,可以自行脑内想像来进行故事,这个作法能让你找到不少漏洞,例如说线索衔接上的错误,或是对玩家可能提出的质问能先有心理准备,预设好几个可能的结局,从最好的到最差的。

虽然知道许多人是因为看了克苏鲁的呼唤跑团影片,才会想要试着写悬疑探索型的剧本,不过很抱歉要说的是,个人其实对克苏鲁神话并不感兴趣,事实上,现实存在的文化与神话传说中,能取用的题材比虚构作品的克苏鲁神话远远要多,更深去挖掘文献会发现,其实存在于真实神话中的黑暗面以及可发挥的点,可以说是无穷无尽。这次就以之前带过的剧本来当范例,并且在最后也欢迎玩家提供灵感来试着做剧本的延伸讨论。

在阅读文献时看到了几则有趣的记载。

\begin{quote}
《东晋干宝·搜神记》
鲁,牛哀,得疾,七日化而为虎,形体变易,爪牙施张。其兄启户而入,搏而食之。方其为人,不知其将为虎也;方有为虎,不知其常为人也。
\end{quote}

另外在卷十二有以下的内容。

\begin{quote}
江,汉之域,有“貙人”,其先,廪君之苗裔也,能化为虎。长沙所属蛮县东高居民,曾作槛捕虎,槛发,明日众人共往格之,见一亭长,赤帻,大冠,在槛中坐。因问“君何以入此中?”亨长大怒曰:“昨忽被县召,夜避雨,遂误入此中。急出我。”曰:“君见召,不当有文书耶?”即出怀中召文书。于是即出之。寻视,乃化为虎,上山走。或云:“貙,虎化为人,如着紫葛衣,其足无踵,虎,有五指者,皆是貙。”
\end{quote}

\begin{quote}
《宋李昉等·太平广记》
唐长安年中,郴州佐史因病而为虎。将啖其嫂,村人擒获,乃佐史也。虽形未全改,而尾实虎矣。因系树数十日,还复为人。长史崔玄简亲问其故。佐史云:“初被一虎引见一妇人,盛服。诸虎恒参集,各令取当日之食。时某新预虎列,质未全,不能别觅他人,将取嫂以供,遂为所擒。今虽作虎不得,尚能其声耳。”简令试之,史乃作虎声,震骇左右,檐瓦振落。
\end{quote}

从这几则文献中可以归纳几个要点。

\begin{enumerate}
\item 大病后成虎。
\item 在江汉之域,也就是湖北北部汉水流域,曾经有一个能化成虎的少数民族。
\item 廪君是湖北巴人的部落领袖,死后魂魄化为白虎。
\end{enumerate}

这次的剧本是用于单元剧团《东京怪奇事件簿》中,玩家扮演的是身怀绝技但因为犯罪(或被陷害)而入狱,在与政府交换条件后,以替政府处理台面上无法公开的各种超自然事件为交换,而获得自由身的角色。

基于以上文献的纪录,把关键人物设定为身在日本的中国人,在一场大病后而获得变身为虎的能力。在整理归纳了几个合理可能性后,将剧本设定为一对湖北人兄妹,偷渡到日本打工赚钱,妹妹王翠华在中菜馆当服务生,但因貌美而被仲介黑工的帮派份子调戏,意外坠楼身亡;其兄王强愤而去讨公道,但不意外的换来一顿痛揍,王强重伤回家后大病卧床数日,在气愤交加的状况下,唤醒了体内深处那化虎的能力。在解释上就是一种“返祖现象”,其久远的祖先有化虎能力。

在王强获得能异变成半人半虎怪物的能力后,再度的回到该黑帮的据点,用爪牙撕碎了害死其妹的凶手,接着从窗户逃窜出去。

以上是所谓的“发生了什么?”,而再来是玩家角色的介入角度,角色们看到了什么?

专责处理超自然现象的政府组织,收到了东京都内出现大型肉食兽的消息而前来调查。横尸于自己帮派办公室的流氓首领,现场是野兽般凌乱的破坏。在东京都内有大型野兽肆虐?却又能神秘地消失。

调查了尸体与房间内部的破坏痕迹后,可以判断是大型肉食兽的攻击痕迹,虽然说想要调监视影像,但小组织多有不法情事,并没有设置会留下纪录的摄影机;可以推测凶手是从窗户离开的,窗户外的窄巷没有足迹,水泥墙壁却有延伸往上的爪痕。

设计剧本时在确定发生了什么事后,可以开始想像其他相应会发生的事,例如说被害者有没有抵抗?发出的动静可以询问附近的人来确定时间点。接着询问附近可否有目击者?

帮派的秘书提供了重要的证词:快八点的时候,有一个二十出头,身高170左右的短发男子来找社长,口音奇怪穿着T恤牛仔裤,后来我听到吵闹声,重物撞击,枪声,玻璃声,敲了敲门后进入发现现场,立刻就报警。

玩家角色能从这个证词知道,这个青年即使不是凶手,也与事件有很大的关系。

从现场的采证中,发现了野兽的鬃毛、射击的弹壳、与遇害者身上衣着不符的布片。另外水泥墙上的爪痕是四爪。

这些线索提供了几个暗示。现场确实出现野兽,有弹壳没弹头表示有东西中枪,不符的布片暗示有人在此处撕破衣服,而玩家也可以依照布料的线索去追查,另外四爪的特征则是暗示着野兽并非正常野兽,拇指能开握是人类的特征。

而在附近闲晃的十岁小孩提供了一个证词:看到一个两公尺以上,从天而降的外国摔角手。落地没有声音,戴了虎面人的面具。

对小孩的逻辑来说,有虎脸的人就是虎面人也就是摔角手。

在放置了足够的线索后,可以来思考一下,当王强作案后会去哪里?玩家角色要去哪里找他?而谁要来告诉玩家这个情报?

为此,在剧本中加入了一个设定“龙之街”,是一个处于大楼间的窄巷所集合的空间,类似西区狮子林等区域但又更阴暗些,又或者像九龙城寨。这个地方是在日本中国人的避风港,有着自己的行事规则与法律。

而提供这个情报的可以是很多不同的人,也许是当地警局处理中国人事务的专员,在这剧本中是由王翠华的同事江琪所告知。

接着发生什么事,就端看玩家角色们的行动。玩家会有两个行动方针,一个是怎么找到人,一个是弄清楚发生什么事。依照这样的思路可以再放置更多细节,例如说与龙之街主事者黄老的对谈交涉,或是煽动当地黑帮与中国黑帮火拼,又或者是偷偷潜入龙之街将王强逮出,也可能发生劝说王强加入组织而受政府庇护的状况。TRPG就是因为这些不确定性与弹性才有趣,不是吗?

最后的总结归纳,这个剧本是由几个碎片所形成的,一个是古籍中“人化为虎”的传说,另一个是觉得“在邻接的大楼间跳跃很帅”的想法,因此设计了这样的剧本与场景。藉由发想与搜寻资料,将几个碎片用名为合理化的浆糊做为连结,仔细的思考说“这样合理吗?”而“接下来可能合理的发展是什么?”,将一幕幕连结润饰后,就能成为一个剧本。

当然这样的基础也能发展出更多情节,不过以原来一日单元剧的团来说,简单的有事发生且查明解决,就很丰富了。

\end{document}
